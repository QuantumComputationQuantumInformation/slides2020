\documentclass{beamer}
%\usepackage{xspace}
\usepackage{amsmath,amssymb}
\usepackage{graphicx}
%\usepackage{svg}
%\usepackage{pgfpages}
%\pgfpagesuselayout{4 on 1}[a4paper,border shrink=5mm,landscape]
%\usepackage{psfrag}
%\usepackage[usenames,dvipsnames]{xcolor}
\usepackage{braket}
\usepackage{qcircuit}
\usepackage{tikz}
\usepackage{tikz-3dplot}
\usetikzlibrary{circuits.logic.US}
\usetikzlibrary{graphs}
\usetikzlibrary{datavisualization}
\usetikzlibrary{datavisualization.formats.functions}
\usepackage{pgfplotstable}
\usepgfplotslibrary{patchplots}

\setbeamercovered{transparent}

\usetheme{Pittsburgh}
%\usetheme{default}

\setbeamertemplate{sidebar right}{}
\setbeamertemplate{footline}[frame number]
%\usefonttheme{professionalfonts}

%\usepackage{sansmathaccent}
%\usepackage{bm}

%\usepackage{unicode-math}
%%\setmainfont[SlantedFont={Latin Modern Roman Slanted},SlantedFeatures={Color=000000},
%%  SmallCapsFont={TeX Gyre Termes},SmallCapsFeatures={Letters=SmallCaps}]{XITS}
%\setmathfont[math-style=ISO,sans-style=upright]{XITS Math}
%\setmathfont[range={\mathcal,\mathbfcal}]{Latin Modern Math}

\usepackage{sfmath}

%\mathversion{sans}

\newcommand{\Tr}{\mathsf{Tr}}

\definecolor{redorange}{rgb}{1.0, .25, .25}
\definecolor{citation}{rgb}{.1, 0.8, .35}
\newcommand\emm[1]{\textcolor{redorange}{{#1}}}
\newcommand\numc[1]{\textcolor{citation}{{\bf #1}}}

%\newcommand\bm[1]{{\mbox{\boldmath $#1$}}}
\newcommand\bm[1]{{\mathbf{#1}}}
%\newcommand\bm[1]{{\bf #1}}
%\newcommand\bm[1]{\ensuremath{\boldsymbol{#1}}}
%\newcommand\bm[1]{{\textbf{\it #1}}}

\title{Universality of quantum circuit}
\author{Ryuhei Mori}
%\institute{$\vcenter{\hbox{\includegraphics[width=30pt]{ELC_logo}}}$ Postdoctoral Fellow of ELC\\ $\vcenter{\hbox{\includegraphics[width=20pt]{titech_logo}}}$ Tokyo Institute of Technology}
\institute{Tokyo Institute of Technology}
%\date{21, Feb, 2019}



\begin{document}
\begin{frame}[plain]
\maketitle
\end{frame}




\begin{frame}{Universality of a quantum circuit}
\begin{theorem}[Universality of finite gate set]
For any unitary matrix $U\in L(\mathbb{C}^{2^n})$ and $\emm{\epsilon} >0$,
there is a quantum circuit with \emm{$X,\,Y,\,Z,\,H,\,S,\,T,\,\mathsf{CNOT}$} gates computing $\widetilde{U}$
satisfying $\|U-\widetilde{U}\|<\emm{\epsilon}$.
\end{theorem}
\begin{proof}
\begin{enumerate}
\setlength{\itemsep}{1em}
\item Any unitary matrix can be decomposed to a product of \emm{two-level unitary matrices}. {\color{green}{Done}}
\item Any two-level unitary matrix can be decomposed to a product of \emm{controlled-unitary gates}. {\color{green}{Done}}
\item Any controlled-untary gate can be decomposed to a product of \emm{CNOT and arbitrary single-qubit gates}.
\item Any single-qubit gate can be approximated by $X,\,Y,\,Z,\,H,\,S$ and $T$.
\end{enumerate}
\end{proof}
\end{frame}

\begin{frame}{Special unitary group}
\begin{itemize}
\setlength{\itemsep}{2em}
\item $\mathsf{U}(n) := \text{the set of $n\times n$ unitary matrices}.$
\item $\mathsf{SU}(n) := \text{the set of $n\times n$ unitary matrices $U$ with $\emm{\det(U)=1}$}.$
\item $\mathsf{U}(n)$ and $\mathsf{SU}(n)$ are groups.
\item For $U\in\mathsf{SU}(n)$ and $V\in\mathsf{U}(n)$, $VUV^\dagger\in\mathsf{SU}(n)$.
\item For $V\in\mathsf{U}(n)$ and $W\in\mathsf{U}(n)$, $VWV^\dagger W^\dagger\in\mathsf{SU}(n)$.
\end{itemize}
\end{frame}

\if0
\begin{frame}{Decomposition to two-level unitary matrix 1/3}
\begin{equation*}
U=
\begin{bmatrix}
u_{1,1}&u_{1,2}&u_{1,3}&u_{1,4}\\
\emm{u_{2,1}}&u_{2,2}&u_{2,3}&u_{2,4}\\
u_{3,1}&u_{3,2}&u_{3,3}&u_{3,4}\\
u_{4,1}&u_{4,2}&u_{4,3}&u_{4,4}
\end{bmatrix}
\end{equation*}
If $u_{2,1}= 0$, we skip this step.

If $u_{2,1}\ne 0$, apply the two-level unitary matrix
\begin{equation*}
V_1=
\frac1z
\begin{bmatrix}
u_{1,1}^*&u_{2,1}^*&0&0\\
\emm{\mathsf{e}^{i\alpha}}u_{2,1}&-\emm{\mathsf{e}^{i\alpha}}u_{1,1}&0&0\\
0&0&z&0\\
0&0&0&z
\end{bmatrix}
\end{equation*}
for $z:=\sqrt{|u_{1,1}|^2+|u_{2,1}|^2}$.
\end{frame}
\fi


\begin{frame}{Controlled-unitary}
\begin{theorem}
Any controlled-unitary gate can be decomposed to a product of \emm{CNOT and arbitrary single-qubit gates}.
\end{theorem}
\begin{proof}
\begin{enumerate}
\setlength{\itemsep}{2em}
\item Controlled-$\emm{\mathsf{U}(2)}$ with \emm{single} controlled qubit.
\item Controlled-$\emm{\mathsf{SU}(2)}$ with \emm{$n$} controlled qubits.
\item Controlled-$\emm{\mathsf{U}(2)}$ with \emm{$n$} controlled qubits.
\end{enumerate}
\end{proof}
\end{frame}

\begin{frame}{Decomposition of single qubit unitary}
\begin{lemma}
Any single qubit unitary $U$, there is single qubit unitary matrices $A,\,B,\,C$ such that
$\emm{ABC=I}$ and $\emm{\mathsf{e}^{i\alpha}AXBXC=U}$.
\end{lemma}

\vspace{2em}
From this lemma,
\[
\Qcircuit @C=2em @R=2em {
& \ctrl{1} & \qw\\
& \gate{U} & \qw\\
}
\quad = \quad
\Qcircuit @C=2em @R=2em {
& \gate{\begin{bmatrix}1&0\\0&\mathsf{e}^{i\alpha}\end{bmatrix}}  & \ctrl{1} & \qw      & \ctrl{1} & \qw & \qw\\
& \gate{C} & \targ    & \gate{B} & \targ    & \gate{A} & \qw\\
}
\]
\end{frame}

\begin{frame}{Decomposition of single qubit unitary}
\begin{lemma}
Any single qubit unitary $U$, there is single qubit unitary matrices $A,\,B,\,C$ and $\alpha\in\mathbb{R}$ such that
$\emm{ABC=I}$ and $\emm{\mathsf{e}^{i\alpha}AXBXC=U}$.
\end{lemma}
\begin{proof}
For any $2\times 2$ unitary matrix, there exist $\alpha,\,\beta,\,\gamma,\,\delta\in\mathbb{R}$ such that
\begin{equation*}
U=\mathsf{e}^{i\alpha}R_Z(\beta)R_Y(\gamma)R_Z(\delta)
\end{equation*}
Let $A:= R_Z(\beta)R_Y(\gamma/2)$, $B:=R_Y(-\gamma/2)R_Z(-(\beta+\delta)/2)$, $C:=R_Z((\delta-\beta)/2)$.
Then, $ABC=I$.
Since $R_Y(\theta)X = XR_Y(-\theta)$ and $R_Z(\theta)X=XR_Z(-\theta)$,.
\begin{align*}
A\emm{X}B\emm{X}C &=
R_Z(\beta)R_Y(\gamma/2)\emm{X}R_Y(-\gamma/2)R_Z(-(\beta+\delta)/2)\emm{X}R_Z((\delta-\beta)/2)\\
&= R_Z(\beta)R_Y(\gamma/2)R_Y(\gamma/2)R_Z((\beta+\delta)/2)R_Z((\delta-\beta)/2)\\
&= R_Z(\beta)R_Y(\gamma)R_Z(\delta) = \mathsf{e}^{-i\alpha}U.
\end{align*}
\end{proof}
\end{frame}

\begin{frame}{Decomposition of single qubit unitary}
\begin{align*}
&\mathsf{e}^{i\alpha}R_Z(\beta)R_Y(\gamma)R_Z(\delta)\\
&=
\mathsf{e}^{i\alpha}
\begin{bmatrix}
e^{-i\beta/2}&0\\
0&\mathsf{e}^{i\beta/2}
\end{bmatrix}
\begin{bmatrix}
\cos(\gamma/2)&\sin(\gamma/2)\\
\sin(\gamma/2)&\cos(\gamma/2)
\end{bmatrix}
\begin{bmatrix}
e^{-i\delta/2}&0\\
0&\mathsf{e}^{i\delta/2}
\end{bmatrix}\\
&=
\mathsf{e}^{i\alpha}
\begin{bmatrix}
\mathsf{e}^{i(-\beta/2-\delta/2)}\cos(\gamma/2)&\mathsf{e}^{i(-\beta/2+\delta/2)}\sin(\gamma/2)\\
\mathsf{e}^{i(+\beta/2-\delta/2)}\sin(\gamma/2)&\mathsf{e}^{i(+\beta/2+\delta/2)}\cos(\gamma/2)
\end{bmatrix}
\end{align*}
\begin{align*}
U&=\begin{bmatrix} a&b\\c&d\end{bmatrix}
\end{align*}
\begin{align*}
|a|^2+|b|^2 &= 1&
|c|^2+|d|^2 &= 1\\
a^*c + b^*d &= 0
\end{align*}
\end{frame}

\begin{frame}{Controlled-unitary}
\begin{theorem}
Any controlled-unitary gate can be decomposed to a product of \emm{CNOT and arbitrary single-qubit gates}.
\end{theorem}
\begin{proof}
\begin{enumerate}
\setlength{\itemsep}{2em}
\item Controlled-$\emm{\mathsf{U}(2)}$ with \emm{single} controlled qubit. {\color{green}{Done}}
\item Controlled-$\emm{\mathsf{SU}(2)}$ with \emm{$n$} controlled qubits.
\item Controlled-$\emm{\mathsf{U}(2)}$ with \emm{$n$} controlled qubits.
\end{enumerate}
\end{proof}
\end{frame}



\begin{frame}{Special unitary group and rotation}
For a real unit vector $\hat{n}=[n_X\ n_Y\ n_Z]$, let
\begin{equation*}
R_{\hat{n}}(\theta):=\cos\frac{\theta}2I -i\sin\frac{\theta}2(n_XX+n_YY+n_ZZ).
\end{equation*}
For any $U\in\mathsf{U}(2)$, there exist $\alpha,\,\theta\in\mathbb{R}$ and a real unit three-dimensional vector $\hat{n}$ such that
$U=\mathsf{e}^{i\alpha}R_{\hat{n}}(\theta)$.

\vspace{2em}
$U\in\mathsf{U}(2)$ is in $\mathsf{SU}(2)$ iff $\Tr(U)\in\mathbb{R}$ since two eigenvalues of $U\in\mathsf{SU}(2)$ are in the form $\{\mathsf{e}^{i\theta}, \mathsf{e}^{-i\theta}\}$.

\vspace{2em}
$U\in\mathsf{U}(2)$ is in $\mathsf{SU}(2)$ iff $\emm{U=R_{\hat{n}}(\theta)}$ for some $\theta\in\mathbb{R}$ and rear unit vector $\widehat{n}\in\mathbb{R}^3$.
\end{frame}

\begin{frame}{Special unitary group and group commutator}
\small
\begin{theorem}
For any $U\in\mathsf{SU}(2)$, there exist $V,\,W\in\mathsf{SU}(2)$ such that $U = VWV^\dagger W^\dagger$.
\end{theorem}
\begin{proof}

\vspace{-2em}
\begin{align*}
&R_Z(\theta)R_X(\theta)R_Z(\theta)^\dagger R_X(\theta)^\dagger
= R_Z(\theta)R_X(\theta)R_Z(-\theta)R_X(-\theta)\\
%&= R_Z(\theta)R_X(\theta)R_Z(-\theta)R_X(-\theta)\\
&= \left[\cos\frac{\theta}2 I - i\sin\frac{\theta}2 Z\right]
\left[\cos\frac{\theta}2 I - i\sin\frac{\theta}2 X\right]
\left[\cos\frac{\theta}2 I + i\sin\frac{\theta}2 Z\right]
\left[\cos\frac{\theta}2 I + i\sin\frac{\theta}2 X\right]\\
&= \left[\cos^4\frac{\theta}2+2\cos^2\frac{\theta}2\sin^2\frac{\theta}2-\sin^4\frac{\theta}2\right] I + \dotsb\\
%&= \left[\cos^2\frac{\theta}2(\cos^2\frac{\theta}2+2\sin^2\frac{\theta}2)-\sin^4\frac{\theta}2\right] I + \dotsb\\
&= \left[1-2\sin^4\frac{\theta}2\right] I + \dotsb
=R_{\widehat{n}_\theta}(\emm{\varphi})
%&= \left[\cos^2\frac{\theta}2 I - i\sin\frac{\theta}2\cos\frac{\theta}2(Z+X) - \sin^2\frac{\theta}2 ZX\right]\\
%&\cdot \left[\cos^2\frac{\theta}2 I + i\sin\frac{\theta}2\cos\frac{\theta}2(Z+X) - \sin^2\frac{\theta}2 ZX\right]
%&= \left[\cos^2\frac{\theta}2 I - \sin^2\frac{\theta}2 ZX\right]^2 - i\sin\frac{\theta}2\cos\frac{\theta}2(Z+X) \right]\\
\end{align*}
\emm{$\cos\frac{\varphi}2 = 1-2\sin^4\frac{\theta}2$}.
%$\sin^2\frac{\varphi}2 = 4\sin^4\frac{\theta}2 - 4\sin^8\frac{\theta}2$.
%$\sin\frac{\varphi}2 = 2\sin^2\frac{\theta}2\sqrt{1 - \sin^4\frac{\theta}2}$.
For some $S\in \mathsf{U}(2)$ and $\varphi\in\mathbb{R}$, $U=SR_{\widehat{n}_\theta}(\varphi)S^\dagger$.
For $V:= SR_Z(\theta)S^\dagger$ and $W:= SR_X(\theta)S^\dagger$,
$U=VWV^\dagger W^\dagger$.
\end{proof}
\end{frame}

\begin{frame}{Group commutator and controlled-unitary}
\begin{theorem}
For any $U\in\emm{\mathsf{SU}(2)}$, controlled-$U$ gate with $n$ controlled qubits can be realized by $O(n^2)$ CNOT and arbitrary single-qubit gates without ancillas (working qubits).
\end{theorem}
\begin{proof}
\small
Induction on $n$.
For the \emm{group commutator decomposition} $U=VWV^\dagger W^\dagger$ using $V,\,W\in \mathsf{SU}(2)$,
\[
\Qcircuit @C=2em @R=1em {
& \ctrl{6} & \qw      & \ctrl{6} & \qw     & \qw\\
& \ctrl{5} & \qw      & \ctrl{5} & \qw     & \qw\\
& \ctrl{4} & \qw      & \ctrl{4} & \qw     & \qw\\
& \qw      & \ctrl{3} & \qw      & \ctrl{3}& \qw\\
& \qw      & \ctrl{2} & \qw      & \ctrl{2}& \qw\\
& \qw      & \ctrl{1} & \qw      & \ctrl{1}& \qw\\
& \gate{W^\dagger} & \gate{V^\dagger} & \gate{W} & \gate{V}& \qw
}
\]
$S_n = 4 S_{n/2} = 4^{\log n}S_1 = O(\emm{n^2})$.
\end{proof}
\end{frame}

\begin{frame}{Controlled-unitary}
\begin{theorem}
Any controlled-unitary gate can be decomposed to a product of \emm{CNOT and arbitrary single-qubit gates}.
\end{theorem}
\begin{proof}
\begin{enumerate}
\setlength{\itemsep}{2em}
\item Controlled-$\emm{\mathsf{U}(2)}$ with \emm{single} controlled qubit. {\color{green}{Done}}
\item Controlled-$\emm{\mathsf{SU}(2)}$ with \emm{$n$} controlled qubits. {\color{green}{Done}}
\item Controlled-$\emm{\mathsf{U}(2)}$ with \emm{$n$} controlled qubits.
\end{enumerate}
\end{proof}
\end{frame}

\begin{frame}{Controlled-$U(2)$ with $n$ controlled qubits}
For any $U\in\mathsf{U}(2)$, there exists $V\in\mathsf{SU}(2)$ and $\alpha\in\mathbb{R}$ such that $U=\mathsf{e}^{i\alpha}V$.

\[
\Qcircuit @C=2em @R=1em {
& \ctrl{6} & \qw\\
& \ctrl{5} & \qw\\
& \ctrl{4} & \qw\\
& \ctrl{3} & \qw\\
& \ctrl{2} & \qw\\
& \ctrl{1} & \qw\\
& \gate{U} & \qw
}
=
\Qcircuit @C=2em @R=1em {
& \ctrl{6} & \ctrl{6} & \qw\\
& \ctrl{5} & \ctrl{5} & \qw\\
& \ctrl{4} & \ctrl{4} & \qw\\
& \ctrl{3} & \ctrl{3} & \qw\\
& \ctrl{2} & \ctrl{2} & \qw\\
& \ctrl{1} & \ctrl{1} & \qw\\
& \gate{V} & \gate{\mathsf{e}^{i\alpha}I} & \qw
}
=
\Qcircuit @C=2em @R=1em {
& \ctrl{6} & \ctrl{5} & \qw\\
& \ctrl{5} & \ctrl{4} & \qw\\
& \ctrl{4} & \ctrl{3} & \qw\\
& \ctrl{3} & \ctrl{2} & \qw\\
& \ctrl{2} & \ctrl{1} & \qw\\
& \ctrl{1} & \gate{\begin{bmatrix}1&0\\0&\mathsf{e}^{i\alpha}\end{bmatrix}} & \qw\\
& \gate{V} & \qw & \qw
}
\]
$A_n = S_n + A_{n-1} = O(n^3)$
\end{frame}

\begin{frame}{Controlled-unitary}
\begin{theorem}
Any controlled-unitary gate can be decomposed to a product of \emm{CNOT and arbitrary single-qubit gates}.
\end{theorem}
\begin{proof}
\begin{enumerate}
\setlength{\itemsep}{2em}
\item Controlled-$\emm{\mathsf{U}(2)}$ with \emm{single} controlled qubit. {\color{green}{Done}}
\item Controlled-$\emm{\mathsf{SU}(2)}$ with \emm{$n$} controlled qubits. {\color{green}{Done}}
\item Controlled-$\emm{\mathsf{U}(2)}$ with \emm{$n$} controlled qubits. {\color{green}{Done}}
\end{enumerate}
\end{proof}
\end{frame}

\begin{frame}{Universality of a quantum circuit}
\begin{theorem}[Universality of finite gate set]
For any unitary matrix $U\in L(\mathbb{C}^{2^n})$ and $\emm{\epsilon} >0$,
there is a quantum circuit with \emm{$X,\,Y,\,Z,\,H,\,S,\,T,\,\mathsf{CNOT}$} gates computing $\widetilde{U}$
satisfying $\|U-\widetilde{U}\|<\emm{\epsilon}$.
\end{theorem}
\begin{proof}
\begin{enumerate}
\setlength{\itemsep}{1em}
\item Any unitary matrix can be decomposed to a product of \emm{two-level unitary matrices}. {\color{green}{Done}}
\item Any two-level unitary matrix can be decomposed to a product of \emm{controlled-unitary gates}. {\color{green}{Done}}
\item Any controlled-untary gate can be decomposed to a product of \emm{CNOT and arbitrary single-qubit gates}. {\color{green}{Done}}
\item Any single-qubit gate can be approximated by $X,\,Y,\,Z,\,H,\,S$ and $T$.
\end{enumerate}
\end{proof}
\end{frame}

\begin{frame}{Approximation of a single-qubit gate is sufficient}
\begin{theorem}
Any single-qubit gate can be approximated by $X,\,Y,\,Z,\,H,\,S$ and $T$.
\end{theorem}
\small
%For $A\in L(\mathbb{C}^n, \mathbb{C}^m)$,
%\begin{align*}
%\|A\| := \max_{\ket{\psi}\in\mathbb{C}^n\colon \braket{\psi|\psi}=1}\sqrt{\bra{\psi}A^\dagger A\ket{\psi}}
%\end{align*}
%$\|A\|$ is the largest singular value of $A$.
Assume that this theorem holds. For $A\in L(\mathbb{C}^d)$, Let $\|A\|$ be the \emm{spectral norm}, which satisfies  $\|UAV\| = \|A\|$ for any unitary matrices $U$ and $V$.
Assume $\|U_i-V_i\|\le\epsilon$ for $i=1,\dotsc,m$.
\begin{align*}
&\|U_mU_{m-1}\dotsm U_1 - V_mV_{m-1}\dotsm V_1\|\\
&=\left\|\sum_{i=1}^{m} (U_m\dotsm U_i V_{i-1}\dotsm V_1  - U_m\dotsm U_{i+1}V_{i}\dotsm V_1)\right\|\\
&\le\sum_{i=1}^{m} \left\|U_m\dotsm U_i V_{i-1}\dotsm V_1  - U_m\dotsm U_{i+1}V_{i}\dotsm V_1\right\|\\
&=\sum_{i=1}^{m} \left\|U_m\dotsm U_{i+1}(U_i-V_i)V_{i-1}\dotsm V_1\right\|
=\sum_{i=1}^{m} \left\|U_i-V_i\right\|\le m\epsilon.
\end{align*}
\end{frame}

\begin{frame}{Universality of $X,Y,Z,H,S,\emm{T}$}
\centering

\tdplotsetmaincoords{70}{120}
\begin{tikzpicture}[tdplot_main_coords, scale=2]

%\shade[ball color = lightgray, opacity = 0.5] (0,0,0) circle (1cm);

\tdplotsetrotatedcoords{20}{80}{0}
\draw[very thick, tdplot_rotated_coords] (0,0,0) circle (1);
%\begin{scope}[canvas is zx plane at y=0]
\draw[dashed, thick] (0,0,0) circle (1);
%\end{scope}

\draw[-stealth] (0,0,0) -- (2.00,0,0) node[below left] {$X$};
\draw[-stealth] (0,0,0) -- (0,1.80,0) node[below right] {$Y$};
\draw[-stealth] (0,0,0) -- (0,0,1.50) node[above] {$Z$};

%\filldraw (0,0,0) circle (1pt) node[right] {$I/2$};
\filldraw (1,0,0) circle (1pt);
\filldraw (-1,0,0) circle (1pt);
\filldraw (0,0,1) circle (1pt);
\filldraw (0,0,-1) circle (1pt);
\filldraw (0,1,0) circle (1pt);
\filldraw (0,-1,0) circle (1pt);
\end{tikzpicture}
\end{frame}

\begin{frame}{Universality of $X,Y,Z,H,S,\emm{T}$}
$T\cong R_Z(\pi/4)$.
$HTH\cong R_X(\pi/4)$.
\begin{align*}
R_Z(\pi/4) R_X(\pi/4) &=
\left[ \cos\frac{\pi}8I - i\sin\frac{\pi}8 Z\right]
\left[ \cos\frac{\pi}8I - i\sin\frac{\pi}8 X\right]\\
&= \cos^2\frac{\pi}8I - i\sin\frac{\pi}8\left[\cos\frac{\pi}8(X+Z)+\sin\frac{\pi}8 Y\right]\\
&=: \cos\frac{\eta}2I - i\sin\frac{\eta}2\left(n_x X + n_Y Y + n_Z Z\right)\\
&= R_{\widehat{n}}(\eta)
\end{align*}
where $\eta$ satisfying $\cos(\eta/2) = \cos^2 (\pi/8)$ and $\widehat{n}$ is a unit vector along with $(\cos\frac{\pi}8,\sin\frac{\pi}8,\cos\frac{\pi}8)$.
Here, $\eta$ is an \emm{irrational multiple of $\pi$}.
$HR_{\widehat{n}}(\eta)H =R_{\widehat{m}}(\eta)$ where $\widehat{m}$ is a unit vector along with $(\cos\frac{\pi}8,-\sin\frac{\pi}8,\cos\frac{\pi}8)$.

\vspace{1em}
$U=\mathsf{e}^{i\alpha} R_{\widehat{n}}(\beta) R_{\widehat{m}}(\gamma) R_{\widehat{n}}(\delta)$.
\end{frame}

\begin{frame}{Universality of a quantum circuit}
\begin{theorem}[Universality of finite gate set]
For any unitary matrix $U\in L(\mathbb{C}^{2^n})$ and $\emm{\epsilon} >0$,
there is a quantum circuit with \emm{$X,\,Y,\,Z,\,H,\,S,\,T,\,\mathsf{CNOT}$} gates computing $\widetilde{U}$
satisfying $\|U-\widetilde{U}\|<\emm{\epsilon}$.
\end{theorem}
\begin{proof}
\begin{enumerate}
\setlength{\itemsep}{1em}
\item Any unitary matrix can be decomposed to a product of \emm{two-level unitary matrices}. {\color{green}{Done}}
\item Any two-level unitary matrix can be decomposed to a product of \emm{controlled-unitary gates}. {\color{green}{Done}}
\item Any controlled-untary gate can be decomposed to a product of \emm{CNOT and arbitrary single-qubit gates}. {\color{green}{Done}}
\item Any single-qubit gate can be approximated by $X,\,Y,\,Z,\,H,\,S$ and $T$. {\color{green}{Done}}
\end{enumerate}
\end{proof}
\end{frame}

\begin{frame}{Solovay--Kitaev theorem}
\begin{theorem}
Let $\{U_1,\dotsc,U_k\}$ be a dense subset of $\mathsf{SU}(2)$.
Then, any $U\in\mathsf{SU}(2)$ can be approxmiated with error $\epsilon$ by $\emm{[\log (1/\epsilon)]^c}$ multiplications of $\{U_1,\dotsc,U_k\}$.
\end{theorem}
\end{frame}

\if0
\begin{frame}{Assignments (Deadline is Jan.\ 17)}
\small
\begin{enumerate}
\setlength{\itemsep}{1em}
\item Show a decomposition of
\begin{equation*}
\frac12\begin{bmatrix}
1&1&1&1\\
1&i&-1&-i\\
1&-1&1&-1\\
1&-i&-1&i
\end{bmatrix}
\end{equation*}
into a product of two-level unitary matrices.
\item Show a decomposition of two-level unitary
\begin{equation*}
\begin{bmatrix}
1&0&0&0&0&0&0&0\\
0&1&0&0&0&0&0&0\\
0&0&a&0&0&0&0&c\\
0&0&0&1&0&0&0&0\\
0&0&0&0&1&0&0&0\\
0&0&0&0&0&1&0&0\\
0&0&0&0&0&0&1&0\\
0&0&b&0&0&0&0&d\\
\end{bmatrix}
\end{equation*}
into a product of controlled-unitary gates.
\end{enumerate}
\end{frame}
\fi


\end{document}
